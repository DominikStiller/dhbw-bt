Stream processing constitutes a fundamental shift of approach of dealing with data. Embracing the unbounded and unordered nature of streams enables low-latency applications and stream-specific processing techniques like session windowing and continuous pattern recognition. Event stream analytics solutions that natively deal with such data needs to solve the key challenges of time and state. Furthermore, software developers and and architects need to adopt a new mindset when making the switch from bounded batch datasets to incomplete and continuously changing streams.

This thesis highlighted the theoretical and practical aspects of designing and building such event stream analytics solutions that produce correct result even in case of faults while being able to deliver them with low latency at large scales. We successfully designed and implemented an analytics solution for the Helsinki public transportation system which allows us to monitor the current situation in real time. This is just one use case for event stream analytics and we believe that many more areas will profit from these real-time insights.
