Batch data processing has become crucial and powers business-critical operations across industries. However, this traditional style of processing is ill-suited for the continuous streams arising from the digitalization or processes and increasing number of sensors to capture events in a connected world. Streams are inherently unbounded and often unordered, which requires a fundamental shift of approach to guarantee correctness while paving the way for new real-time insight techniques.

The goal of this thesis is to build the understanding required for the design of correct, fault-tolerant, low-latency and scalable event stream analytics solutions whose key challenges are time and state. Then we demonstrate the concepts and design considerations through the implementation of a real-time public transportation analytics solution, which we evaluate to further highlight the practical challenges of building such a solution. The findings and considerations can be applied to a wide range of stream analytics use cases and can help developers and architects to build a stream-native mindset.

\rule{\textwidth}{0.5pt}\vspace{1.5mm}

Batch-Datenverarbeitung ist heutzutage essentiell für businesskritische Operationen in vielen Branchen. Diese traditionelle Art von Datenverarbeitung ist jedoch ungeeignet für die kontinuierlichen Streams, die aufgrund der Digitalisierung von Prozessen und dem Einsatz von Sensoren zur Ereigniserfassung in einer vernetzten Welt immer häufiger auftreten. Streams sind von Natur aus endlos und meist ungeordnet, daher ist ein fundamentalen Paradigmenwechsel notwendig um korrekte Ergebnisse garantieren zu können, aber auch um neue Echtzeit-Verarbeitungsmethoden zu ermöglichen.

Ziel dieser Arbeit ist es, das notwendige Verständnis für die Entwicklung korrekter, fehlertoleranter, niedrig-verzögerter und skalierbarer Stream-Analytics-Lösungen aufzubauen, wobei Zeit und Zustand die größten Herausforderungen darstellen. Außerdem demonstrieren wir die Konzepte und Konzeptionierungs-Überlegungen anhand der Implementierung einer Echtzeit-Analytics-Lösung für den öffentlichen Nahverkehr, welche wir im Folgenden evaluieren, um praktische Probleme aufzuzeigen. Die hier präsentierten Überlegungen können auf viele Stream-Analytics-Usecases angewendet werden und Software-Entwicklern sowie -Architekten helfen, die notwendige Denkweise für Stream-Verarbeitung zu erlangen.
